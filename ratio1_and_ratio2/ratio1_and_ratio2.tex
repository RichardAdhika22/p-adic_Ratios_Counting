%% Standard start of a latex document
\documentclass[letterpaper,12pt]{article}
%% Always use 12pt - it is much easier to read
\usepackage{algorithm}
\usepackage{algpseudocode}
\usepackage{float}

%% AMS mathematics packages - they contain many useful fonts and symbols.
\usepackage{amsmath, amsfonts, amssymb}


%% The geometry package changes the margins to use more of the page, I suggest
%% using it because standard latex margins are chosen for articles and letters,
%% not homework.
\usepackage[paper=letterpaper,left=25mm,right=25mm,top=3cm,bottom=25mm]{geometry}
%% For details of how this package work, google the ``latex geometry documentation''.
\usepackage{amsthm}
\usepackage{geometry}
\geometry{
  left=0.7in,
  right=0.7in,
  top=0.8in,
  bottom=1in
}

\newtheorem{Lemma}{Lemma}
\newtheorem{Theorem}{Theorem}
\usepackage{enumerate}
%%
%% Fancy headers and footers - make the document look nice
\usepackage{fancyhdr} %% for details on how this work, search-engine ``fancyhdr documentation''
% \pagestyle{fancy}
%%
%% This is a little more complicated because we have used `` \\ '' to force a line-break between the name and number.
%%
\newcommand{\Z}{\mathbb{Z}}
\newcommand{\Q}{\mathbb{Q}}
\newcommand{\bigO}{\mathcal{O}}
\newcommand{\GL}{\mathbf {GL}}

\cfoot{Page \thepage} % page in middle

%% These put horizontal lines between the main text and header and footer.
\renewcommand{\footrulewidth}{0.4pt}
%%%

\begin{document}

\begin{center}
    {\bf \Huge Ratio 1 and Ratio 2} \\
    Richard Adhika
\end{center}

\

{\bf \LARGE 1. Introduction}

\

For reasons coming from number theory, we are interested in computing the following two kinds of (closely related) ratios, for different primes $p$: 
Let $t$ and $D$ be integers.  
The first ratio is 
\[
\nu_{p, n} (t, d) = \frac{\#\{M\in \GL_2(\Z/p^n\Z) \vert M \text{ has trace } t \text{ and } \det  D  \bmod p^n\} }{p^{2n-2}(p^2-1)}.
\]

This ratio is really interesting only when the \emph{discriminant} of the characteristic polynomial, i.e., 
$t^2-4D$, is divisible by $p$ (and they get more interesting as the power of $p$ dividing the discriminant gets larger). 
This stabilizes when $n$ gets large (if $p$ doesn't divide the discriminant, they stabilize right away). 
However, we understand these quite well. We are more interested in the ones that come not from counting solutions $\bmod p^n$, but from counting the \emph{reductions  $\bmod p^n$ of solutions in $\Z_p$}. 
More precisely, we define 
\[
\mu_{p, n} (t, d) = \frac{\#\{M\in \GL_2(\Z/p^n\Z) \vert \exists M'\in \GL_2(\Z_p):  M' \text{ has trace } t \text{ and }  \det  D, \text { and } M' \bmod p^n = M \} } 
{p^{2n-2}(p^2-1)}.
\]

Since we cannot really handle full $p$-adic series in a computer code, we introduce another parameter, "kbig" or something (also user-defined), and compute 
\[
\mu_{p, n} (t, d) = \frac{\#\{M\in \GL_2(\Z/p^n\Z) \vert \exists M'\in \GL_2(\Z/p^{kbig}\Z):  M' \text{ has trace } t \text{ and }  \det  D, \text { and } M' \bmod p^n = M \} } 
{p^{2n-2}(p^2-1)}.
\]

Again this stabilizes eventually (as $n$ gets large, but we need $kbig>>n$), and again it is only interesting when a power of $p$ divides the discriminant (when it doesn't, this also stabilizes at $n=1$ and are equal to $\nu_p$). 
We want to know these numbers for various values of $t$, $d$, and $p$.
The problem is that even if you let $p=5$ and $kbig=6$, it is already taking forever to compute 
(using brute force method). 

\vspace{0.6in}

{\bf \LARGE 2. Ratio 1}

\

We first want to try to generate all matrices of the form 
$\begin{pmatrix} a & b \\ c & d \end{pmatrix}$
with $d \equiv t - a \bmod p^n$ and $ad - bc \equiv D \bmod p^n$ 
for $a,b,c,d \in \Z/p^n\Z$.
Note that throughout this file, this notation will be used when referring to a general $2\times2$ matrix.
Here, we have 3 free parameters $a,b$, and $c$ since $d$ depends solely on $a$ for the given $t$.
The naive brute force method is to loop
through $[0, p^n-1]$ for each $a,b,c$ and then check the determinant condition.
This takes $\bigO(p^{3n})$, which is very slow.
Hence, the function \texttt{build\_matrices1} in the file 
\emph{ratio1\_and\_ratio2.ipynb} (as well as all other code discussed in this file) aims to 
generate the matrices more efficiently.

\

For the parameter $a$, we will still loop through $[0, p^n-1]$.
First, fix an $a$.
Let rhs$ = ad-D \bmod p^n$.
Then, for any $b \in [0, p^n-1]$,
the problem turns into solving the equation (for $c$)
\begin{equation}
bc \equiv rhs \bmod p^n .
\end{equation}
If $p \nmid b$, then $b^{-1}$ exists and thus 
we have a unique solution
$c \equiv rhs \cdot b^{-1} \bmod p^n$ (line 6).
We don't need to loop through $c$ for these $b$.
If $p \mid b$, then we have three cases as follows,

\begin{enumerate}
\item $p \nmid rhs$. This immediately tell us that (1) can't have any solution 
if $p \mid b$. This corresponds to the fact that the loop in line 9 won't execute.

\item $v_p(rhs) \geq 1$.
If $v_p(b) > v_p(rhs)$, then (1) also can't have any solution.
Hence, it suffices to check $b$ with $1 \leq v_p(b) \leq v_p(rhs)$ (line 10 with $v_p(b) = i$).
Then, for each $i$, let $b = p^i j$ so that we can rewrite (1) as 
\[
p^i j \cdot c \equiv rhs \bmod p^n,
\]
We then get $c \equiv (rhs / p^i) \cdot j^{-1} \bmod p^{n-i}$, 
for which we lift $c$ to modulo $p^n$.
Also $v_p(rhs) < n$ so the modulo is well-defined.

\item $rhs \equiv 0 \bmod p^n$.
Then, we can set $b=0$ so that all $c \in [0, p^n-1]$ satisfy (1) (line 14-16).
Furthermore, for any $b$, 
all $c$ with $v_p(c) = n-v_p(b)$ satisfy (1) (line 17 - 19). 
\end{enumerate}

\begin{algorithm}[H]
\caption{build\_matrices1}
\begin{algorithmic}[1]
\Function{build\_matrices1}{$n, p, t, D$}
    \State $res \gets$ [ ], $R_n \gets \Z/p^n\Z$    
    \ForAll{$a \in R_n$}
        \State $rhs \gets R_n(D - a(t-a))$
        \State Add all tuples $(a,b,c,d)$ to $res$ such that $c \equiv rhs \cdot b^{-1} \bmod p^n$ with $p \nmid b$. \\
        
        \If{rhs $\neq 0$}
            \For{$i = 1$ to $v_p(rhs)$}
                \For{$b \in R_n$ with $b = p^ij$ with $p \nmid j$}
                    \State Add $(a,b,c,d)$ to $res$ with $c \in Rn$ satisfying $c \equiv rhs \cdot j^{-1} \bmod p^{n-i}$
                \EndFor
            \EndFor
        \Else
            \ForAll{$c \in R$}
                \State Add $(a,0,c,d)$ to $res$
            \EndFor

            \For{$b \in R_n$ with $v_p(b) = i \geq 1$}
                \State Add $(a,b,c,d)$ to $res$ with $c \in R_n$ satisfying $v_p(c) = n-i$
            \EndFor
        \EndIf
    \EndFor
    \State \Return $res$
\EndFunction
\end{algorithmic}
\end{algorithm}

Assuming $n$ is not large, this algorithm costs $\bigO(p^{2n})$,
which is already quite an improvement from the brute force algorithm.
Notice that in the second case, the number of matrices 
with $b = b'$ satisfying $v_p(b') = v$ is the same as that of $b = p^v$
as both corresponding loops in line 9-11 have the same step size.

\

Now, we can modify this algorithm a bit if we only want to \emph{count} the number of matrices
satisfying the numerator condition.
The few points that we obtained so far are:
\begin{itemize}
\item For each $a$, there are $p^{n-1}(p-1)$ $b$'s that are relatively prime to $p$.
Each of these $b$ produces exactly one matrix.
Hence, we have $p^{2n-1}(p-1)$ matrices in total with $p \nmid b$.
\item For each $a$, if rhs has $p$-valuation $k$, then there are 
\[
\sum_{i=1}^k p^{n-i-1} \cdot (p-1) \cdot p^i = k p^{n-1}(p-1)
\]
many solutions with $1 \leq v_p(b) \leq k$.
The term $p^{n-i-1}(p-1)$ comes from the number of $b$
with $v_p(b) = i$
and the term $p^i$ comes from lifting the modulo from $p^{n-i}$ to $p^n$.
\item For each $a$, if rhs $\equiv 0 \bmod p^n$, then there are 
\[
p^n + \sum_{i=1}^{n-1} p^i (p^{n-i} - p^{n-i-1})
= p^n + (n-1) (p^n - p^{n-1})
\]
many solutions with $b = 0$ and $v_p(b) \geq 1$.
\end{itemize}

With these in mind, we then write a function \texttt{ratio1}
that count the number of matrices satisfying the numerator condition in $\bigO(p^n)$.
The pseudocode is as follows,
\begin{algorithm}[H]
\caption{ratio1}
\begin{algorithmic}[1]
\Function{ratio1}{$n, t, D$}
    \State $count \gets p^n \cdot p^{n-1} \cdot (p - 1)$, $R_n \gets \text{Integers}(p^n)$
    \ForAll{$a \in R$}
        \State $rhs \gets R_n(a(t-a) - D)$

        \If{$rhs \neq 0$}
            \State $count \gets count + v_p(b) \cdot p^{n-1} \cdot (p - 1)$
        \Else
            \State $count \gets count + p^n + (n - 1) \cdot (p^n - p^{n-1})$
        \EndIf
    \EndFor
    \State \Return $count$
\EndFunction
\end{algorithmic}
\end{algorithm}

\newpage

{\bf \LARGE 3. Ratio 2}

\

For the second ratio, however, we still need to generate the matrices 
as the reduction from $kbig$ to $n$ may not include every matrix that satisfies the constraint 
in $\Z / p^n\Z$ (which is true if $p^l | t^2-4D$ for some $l \in \mathbb{N}$).
The problem now is that generating and storing all such matrices 
in $\Z/p^{kbig}\Z$ and then reduce each one to mod $p^n$ takes 
up a lot of time and space (e.g. for $p=5$ and $kbig=5$, we have around 11 million matrices
and for $kbig=6$, CoCalc runs out of space).

\

What if we don't actually need to generate all the matrices?
The first observation is that 
\begin{Lemma}
For any matrix $M = \begin{pmatrix} a & b \\ c & d \end{pmatrix} \in \GL_2(\Z/p^{kbig}\Z)$,
if $p \nmid b$, we only need to consider $a, b \in [0, p^n-1]$.
\end{Lemma}
\begin{proof}
Consider two matrices $\begin{pmatrix} a_1 & b_1 \\ c_1 & d_1 \end{pmatrix}$ and
$\begin{pmatrix} a_2 & b_2 \\ c_2 & d_2 \end{pmatrix}$
with $p \nmid b_1, b_2$, 
$b_2 \equiv b_1 \bmod p^n$, and $a_2 \equiv a_1 \bmod p^n$.
Let $rhs = a_1d_1-D \bmod p^{kbig}$.
Then, the modular equation $b_1c_1 \equiv rhs \bmod p^{kbig}$ has a unique solution 
$c_1 \equiv rhs \cdot b_1^{-1} \bmod p^{kbig}$ .
We therefore have
\[
d_2 \equiv t-a_2 \equiv t-a_1 \equiv d_1 \bmod p^n
\] 
and
\[
c_2 \equiv (a_2d_2 - D)b_2^{-1} 
\equiv (a_1d_1 - D)b_1^{-1} 
\equiv c_1 \bmod p^n.
\]
Hence all $a,b \geq p^n$ with $p \nmid b$ will not 
produce any new reduced matrix.
\end{proof}

This means that the reduction does not affect the existence and uniqueness of matrices 
for $b$ with $p \nmid b$.
Therefore, just as the first ratio, we don't need to consider these $b$'s
and that the count for these matrices is $p^{2n-1}(p-1)$.
Next, we have

\

\begin{Theorem}
Let $A$ be the set of all $M \in \GL_2(\Z/p^n\Z)$ 
such that there exists $M' \in \GL_2(\Z/p^{kbig}\Z)$ having 
trace $t$ and determinant $D$, and that $M \equiv M' \bmod p^n$
(the numerator condition).
Let $S \subset A$ with $b = p^v$ and
$S' \subset A$ with $b = p^vu$ for some fixed $u$ with $p \nmid u$
and $v < n$.
Then, $|S| = |S'|$.
\end{Theorem}

\begin{proof}
Let $v \in [1, n-1]$ and $M = \begin{pmatrix}
a & p^v \\ c & d
\end{pmatrix} \in S$.
Define a map $\phi: S \to S'$ by 
\[
\phi(M) = N = \begin{pmatrix}
a & p^vu \\ cu^{-1} & d
\end{pmatrix}.
\]
First, we want to show that this mapping is well-defined 
by showing that $N \in S'$.
By the definition of $M$, there exists $M' \in \GL_2(\Z/p^{kbig}\Z)$
of the form $M' = \begin{pmatrix}
a' & b' \\ c' & d'
\end{pmatrix}$ having trace $t$ and determinant $D$ and
that $M' \equiv M \mod p^n$.
Let 
\[
U = \begin{pmatrix}
1 & 0 \\ 0 & u
\end{pmatrix}, \;
U^{-1} = \begin{pmatrix}
1 & 0 \\ 0 & u^{-1}
\end{pmatrix}.
\]
Let $N' \in \GL_2(\Z/p^{kbig}\Z)$ be defined by 
\[
N' = U^{-1}MU 
= \begin{pmatrix}
a' & b'u \\ c'u^{-1} & d'
\end{pmatrix}.
\]
Then, $N'$ has trace $t$ and determinant $D$, and that 
$b'u \equiv p^v u \bmod p^n$.
Hence, $N = N' \bmod p^n \in S'$.

\

We then want to show that this mapping is injective.
Let $M_1, M_2 \in S$ of the form 
\[
M_1 = \begin{pmatrix}
a_1 & p^v \\ c_1 & d_1
\end{pmatrix}, 
M_2 = \begin{pmatrix}
a_2 & p^v \\ c_2 & d_2
\end{pmatrix}.
\]
Assume that $\phi(M_1) = \phi(M_2)$.
Hence, we have 
\[
\begin{pmatrix}
a_1 & p^v u \\ c_1u^{-1} & d_1
\end{pmatrix} = \begin{pmatrix}
a_2 & p^v u \\ c_2u^{-1} & d_2
\end{pmatrix}
\]  
Since $p \nmid u$ and $p \nmid u^{-1}$, the equality $M_1 = M_2$ follows.

\

Finally, we want to show that the map is surjective.
Let $N = \begin{pmatrix}
a & p^v u \\ cu^{-1} & d
\end{pmatrix} \in S'$ (any matrix in $S'$ can be written in this form).
Let $M = \begin{pmatrix}
a & p^v \\ c & d
\end{pmatrix}$.
If $N'$ is the lift of $N$, then we define 
$M' = U^{-1}N'U$ and a similar argument as before
shows that $M \equiv M' \bmod   p^n$.
Hence $M \in S$ and that $\phi(M) = N$ 
so that $\phi$ is surjective.
Finally, we conclude that $\phi$ is a well-defined bijection 
between $S$ and $S'$ so that $|S| = |S'|$.
\end{proof}

\

By Theorem 1, for ratio 2, we \emph{only} need to 
consider the reduced matrices with $b = p^v$ for $1 \leq v < kbig$ and $b = 0$.
There are still a few points that can be optimized especially with the heavy loops of 
the parameters $b$ and $c$, which is what we are trying to do next.

\

\begin{Theorem}
Let $S \subset \GL_2(\Z / p^{kbig}\Z)$ be the set of all matrices that we have encountered so far in the loop, 
having trace $t$ and determinant $D$. 
Let $M = \begin{pmatrix} a_1 & b_1 \\ c_1 & d_1 \end{pmatrix} \in S$.
Furthermore, let $N = \begin{pmatrix} a_2 & b_2 \\ c_2 & d_2 \end{pmatrix} \in \GL_2(\Z / p^{kbig}\Z)$ 
be the matrix that we are considering
with $N \equiv M \bmod p^n$ and $v_p(b) = v_p(b') = i$.
Then, we don't need to consider all matrices of the form 
$N' = \begin{pmatrix} a_2 & b_2' \\ c_2' & d_2 \end{pmatrix}$ for 
\begin{itemize}
\item $b_2' = b_2$ and $c_2' > c_2$ satisfying the determinant constraint, or
\item $b_2' > b_2$ with $v_p(b_2') = i$ and for any $c_2'$ satisfying the determinant constraint.
\end{itemize}
\end{Theorem}

\

\begin{proof}
To show the first point we want to show that
we don't need to consider further $c$ values once we encounter such $N$.
For any $s \in \mathbb{N}$ and
for any matrix $N' = \begin{pmatrix} a_2 & b_2 \\ c_2 + sp^{kbig-i} & d_2 \end{pmatrix}$,
we have $N' \equiv M' \bmod p^n$ 
where $M' = \begin{pmatrix} a_1 & b_1 \\ c_1 + sp^{kbig-i} & d_1 \end{pmatrix}$.
If either $a_2 > a_1$ or $b_2 > b_1$, it is clear that $M' \in S$.
Otherwise, $c_2 > c_1$ so $M'$ must have also been considered earlier
(in the code itself, this case won't happen since it is enough to consider all matrices with $c \in [0, p^n-1]$).

\

Next, we want to show that if $a_2 > a_1$, then the assumption also holds with $b_2 = p^i$
and that we don't need to consider any matrix of the form 
$N' = \begin{pmatrix} a_2 & sp^i \\ c_2' & d_2 \end{pmatrix}$ with $p \nmid s, s \geq 2$,
and for any $c_2'$ satisfying the determinant constraint.
Let $b_2 = p^i u$ with $p \nmid u$.
We can then write $b_1 = p^i \beta' u$, $c_1 = \gamma' u^{-1}$, and 
$c_1 = \gamma u^{-1}$ so that we have 
\[
\begin{pmatrix} a_2 & p^i u \\ \gamma u^{-1} & d_2 \end{pmatrix}
\equiv \begin{pmatrix} a_1 & p^i \beta' u \\ \gamma' u^{-1} & d_1 \end{pmatrix} \bmod p^n.
\]
Since $p \nmid u, u^{-1}$, the equation above is equivalent to
\[
\begin{pmatrix} a_2 & p^i \\ \gamma & d_2 \end{pmatrix}
\equiv \begin{pmatrix} a_1 & p^i \beta'\\ \gamma' & d_21\end{pmatrix} \bmod p^n.
\]
Since the right matrix is in $S$, the left matrix doesn't give any new reduced matrix
so that the assumption also holds for $b_2 = p^i$.  
Then, for any $s \in \mathbb{N}$ with $p \nmid s$, we have 
\[
\begin{pmatrix} a_2 & p^js \\ \gamma s^{-1} & d_2 \end{pmatrix}
\equiv \begin{pmatrix} a_1 & p^i \beta's\\ \gamma's^{-1} & d_1 \end{pmatrix} \bmod p^n,
\]
for which the right matrix is also in $S$.
Together with the first point, this means that $N'$
have the same reduced form as one of the matrices in $S$.

\

Otherwise, assume that $a_2 = a_1$. 
Then $b_2 > b_1$, so let $b_2 = b_1 + s p^i$ for some $s \in \mathbb{N}$.
Let $\alpha = \frac{a_2d_2-D}{p^i}$.
Then we have 
\[
c_2 \equiv \alpha (b_1 / p^i + s)^{-1} \bmod p^{kbig-i}\; \text{ and } \;
c_1 \equiv \alpha (b_1 / p^i)^{-1} \bmod p^{kbig-i}.
\]
Since $c_2 \equiv c_1 \bmod p^n$, we have
\[
\alpha (b_1 / p^i + s)^{-1} \equiv \alpha (b_1 / p^i)^{-1} \bmod p^n.
\]
Note that $(b_1 / p^i + s)^{-1}$ exists since $v_p(b_2) = i$ by assumption. 
Let $b_2' = b_1+ (s+s')p^i$ with $s' \in \mathbb{N}$.
First, assume that $v_p(\alpha) < n$ so that $s \equiv 0 \bmod p$.
We then have 
\[
c_2' \equiv \alpha (b_1 / p^i + (s+s'))^{-1} \equiv \alpha (b_1 / p^i + s')^{-1} \bmod p^n.
\]
Note that $(b_1/p^i + s')^{-1}$ exists since $b_1/p^i + s' \equiv b_1/p^i + s'+ s' \not\equiv 0 \bmod p$. 
Hence, we have 
\begin{equation}
\begin{pmatrix} a_2 & b_1 + (s+s')p^i \\ c_2' & d_2 \end{pmatrix}
\equiv \begin{pmatrix} a_1 & b_1 + s'p^i \\ c_1' & d_1 \end{pmatrix} \bmod p^n,
\end{equation}
where $c_1' \equiv \alpha (b_1 / p^i + s')^{-1} \bmod p^n$.
If $v_p(\alpha) \geq n$, then (2) also holds with $c_2' \equiv c_1' \equiv 0 \bmod p^n$.
In both cases, we get that the reduced form of $N'$ has been considered earlier.
Hence combined with the first point, we don't need to consider all such $N'$.

\end{proof}

\

By Theorem 2, for a fixed $a$, once we encounter such $N$ in the code,
we can directly move on to another value of $a$.
Before going to the main function \texttt{ratio2}, we will take a look more closely 
at two helper functions: \texttt{ratio2\_adder} and \texttt{ratio2\_adder\_with\_check},
which reduce a lot of repetitions (of $b$ and $c$) to make the code more efficient
by utilizing Theorem 2.
Below is the pseudocode for the function \texttt{ratio2\_adder}.

\begin{algorithm}[H]
\caption{Helper Function 1}
\begin{algorithmic}[1]
\Function{ratio2\_adder}{$p, n, kbig, \text{projset}, a, c_s, exp, R_n$}
    \If{$\big( R_n(a),\; R_n(c_{\text{start}})\big) \in \text{projset}$}
        \State \Return \textbf{false}
    \EndIf
    \State Add $\big( R_n(a),\; R_n(c_{\text{start}}) \big)$ to \text{projset}
    \For{$c$ from $c_{\text{start}} + p^{k_{\max} - \text{exp}}$ to $p^n - 1$ step $p^{k_{\max} - \text{exp}}$}
        \State Add $\big( R_n(a),\; R_n(c) \big)$ to \text{projset}
    \EndFor
    \State \Return \textbf{true}
\EndFunction
\end{algorithmic}
\end{algorithm}

Note that we use this function only for $v_p(b) < n$
as the $b$ entry in all matrices in $projset$ has the same $p$-valuation.
The parameter $projset$ keeps track of the reduced matrices we encounter so far,
$c_s$ denotes the smallest solution of $c$ when solving the determinant constraint, and 
$exp$ denotes $v_p(b)$.
This function returns false if we encounter one reduced matrix 
that has been considered earlier so that the function \texttt{ratio2} can break out of the loop.
For $v_p(b) \geq n$, we have the following similar helper function instead,

\begin{algorithm}[H]
\caption{Helper Function 2}
\begin{algorithmic}[1]
\Function{ratio2\_adder\_with\_check}{$p, n, kbig, \text{lift\_dict}, a, b, c_s, exp, R_n, R_{kbig}$}
    \If{lift\_dict has key $\big( R_n(a),\; R_n(c_s) \big)$}
        \If{the corresponding lifted matrix has $v_p(b) = exp$ or $b \equiv 0 \bmod p^{kbig}$}
            \State \Return \textbf{false}
        \EndIf
    \EndIf
    \State Add $\big( R_n(a),\; R_n(c_s) \big)$ as key and  $\big( R_{kbig}(a),\; R_{kbig}(b) ,\; R_{kbig}(c_s) \big)$ to lift\_dict
    \For{$c$ from $c_s + p^{kbig - exp}$ to $p^n - 1$ step $p^{kbig - exp}$}
        \State Add $\big( R_n(a),\; R_n(c) \big)$ as key and  $\big( R_{kbig}(a),\; R_{kbig}(b) ,\; R_{kbig}(c) \big)$ to lift\_dict
    \EndFor
    \State \Return \textbf{true}
\EndFunction
\end{algorithmic}
\end{algorithm}

The only different thing here is that we now have a dictionary called
$lift\_dict$ that keeps track on the pairing between the reduced matrix
(entries in $\Z/p^n\Z$) and the original matrix (entries in $\Z/p^{kbig}\Z$).
This is done to make sure that Theorem 2 applies as 
we require the assumption that the matrices compared have the same $p$-valuation of their corresponding $b$ entry.
Furthermore, if $b \equiv 0 \bmod p^{kbig}$, this means that we have covered 
all possible reduced form of $c$.

\

The function \texttt{ratio2} uses these two helper functions to 
efficiently count the number of matrices satisfying the numerator of the second ratio.
For $1 \leq v_p(b) \leq n-1$, the function uses \texttt{ratio2\_adder}
and only uses set to keep track of the matrices encountered (only store the reduced matrices),
which can be seen in line 3-17.
On the other hand, for $v_p(b) \geq n$, the function uses 
\texttt{ratio2\_adder\_with\_check} and uses dictionary to keep track of 
both the reduced and original matrices.
It is not easy to assess the asymptotic runtime of this function, but it takes only a few seconds
to compute ratio2 for $p = 5, kmax = 9$.
The pseudocode is as follows, 

\begin{algorithm}[H]
\caption{Ratio2}
\begin{algorithmic}[1]
\Function{ratio2}{$p, kbig, n, t, D$}
    \State $count \gets p^{2n-1}(p-1)$, $R_n \gets \mathbb{Z}/p^n\mathbb{Z}$, $R_{kbig} \gets \mathbb{Z}/p^{kbig}\mathbb{Z}$
    \For{$\exp = 1$ \textbf{to} $n-1$}
        \State $projset \gets \emptyset$
        \For{$a = 0$ \textbf{to} $p^{kbig}-1$}
            \State $rhs \gets R_{kbig}(a \cdot (t-a) - D)$
            \If{$rhs \ne 0$}
                \If{$v_p(rhs) \geq exp$}
                    \State for all $b = p^{exp}u$ with $p \nmid u$, let $c_s \equiv (rhs/p^{exp})u^{-1} \bmod p^{kbig-exp}$
                    \State if $\mathrm{ratio2\_adder}(p,n,kbig,projset,a,c_s,\exp,R_n)$ is \textbf{false}, stop checking other $b$'s
                \EndIf
            \Else
                \State $\mathrm{ratio2\_adder}(p,n,kbig,projset,a,0,\exp,R_n)$
            \EndIf
        \EndFor
        \State $count \gets count + |projset| \cdot \big(p^{\,n-\exp} - p^{\,n-\exp-1}\big)$
    \EndFor \\
    \State $lift\_dict \gets \emptyset$
    \For{$a = 0$ \textbf{to} $p^{kbig}-1$}
        \State $rhs \gets R_{kbig}(a \cdot (t-a) - D)$
        \If{$rhs \ne 0$}
            \For{$\exp = n$ \textbf{to} $v_p(rhs)$}
                \State for all $b = p^{exp}u$ with $p \nmid u$, let $c_s \equiv (rhs/p^{exp})u^{-1} \bmod p^{kbig-exp}$
                \State if $\mathrm{ratio2\_adder\_with\_check}(p,n,kbig,lift\_dict,a,b,c_s,\exp,R_n, R_{kbig})$ is \textbf{false}, stop checking other $b$'s
            \EndFor
        \Else
            \State $\mathrm{ratio2\_adder\_with\_check}(p,n,kbig,lift\_dict,a,0,0,kbig,R_n,R_{kbig})$
        \EndIf
    \EndFor
    \State \Return $count + |\mathrm{lift\_dict}|$
\EndFunction
\end{algorithmic}
\end{algorithm}

% \

\end{document}
